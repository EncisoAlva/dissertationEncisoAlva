\chapter{INTRODUCTION}
\label{ch:intro}

One hundred years after its invention, the Electroencephalogram (EEG) is a popular tool for studying the physiological substrate of behavioral phenomena.
%
The study of brain electrophysiology has led to a rather large corpus of electrical signals that are known to be related to normal and pathological states.
%
For example, the interested reader may refer to the book from Niedermeyer \cite{niedermeyer2011niedermeyer}.

In light of this legacy, an important task for modern Neuroscience is to determine the location of electrical activity of neural origin inside the brain. 
%
For instance, we may take a set of EEG recordings known to be related to a specific cognitive task and then ask which brain region is responsible for creating such electrical activity.
%
This task is known as electrical source localization or electrical source reconstruction.

From this vague definition, the task is ill-posed (in the sense of Hadamard) and impossible to achieve without further information/assumptions.

Electrical Source Imaging is a framework for source reconstruction. Its base model reconstructs a group of dipoles whose electric field is equivalent to that of the neurons.
%
ESI methods are agnostic to the number or extension of electrical sources and require finding a large number of parameters from a linear system.

ESI methods have been used extensively in the literature to study pathological conditions.
%
Asadzadeh et al. report 41 different ESI algorithms described between 1970 and June 2019, most of them used to study a clinical condition.

We explore some of these ESI methods, along with their underlying assumptions and related algorithms.
%
Our objective is to incorporate a particular set of assumptions into the ESI model and derive an ESI method optimized for those assumptions.

The main focus of this work is the usability of the ESI methods in clinical settings.
%
All the methods described in this work are available for public use in the GitHub repository of the author at 
%\url{https://github.com/EncisoAlva/Region-Priors}{github.com/EncisoAlva/Region-Priors}.
https://github.com/EncisoAlva/Region-Priors.
%
These methods are implemented in a format compatible with the Brainstorm toolbox \cite{brainstorm}, making it possible for non-technical users to incorporate them into a data analysis pipeline without writing code.

\section{Contents and Organization}

Chapter \ref{ch:forward} is dedicated to deriving the Forward Problem in ESI from physical principles.
%
At the end of the chapter, it is established that the following equation governs the ESI,
\begin{equation*}
    \Y = \G \SA +\varepsilon
\end{equation*}
with $\Y\in \R^{M\times T}$ encoding the measurements from $M$ point electrodes, $\SA \in \R^{N\times T}$ encoding the electrical activity of $N$ dipoles whose electrical field is equivalent to that of the neurons in the brain, $\G \in \R^{M\times N}$ is a mixing operator, and $\varepsilon \in \R^{M\times N}$ encodes some noise.
%
In this context, it is expected that $M\ll N$.

The Forward Problem in ESI consists of computing $\G$ from anatomical data; some additional assumptions are necessary.
%
%Some commercial software is cited to perform these tasks numerically.
%
The derivation presented in chapter \ref{ch:forward} is standard; it can be found in the paper by Hallez et al. \cite{hallez2007review} or the book by Nunez and Srinivasan \cite{nunez2006electric}.

The objective of chapter \ref{ch:forward} is to present the assumptions required for the model to work.
%
Understanding the derivation of the realistic 4-sphere model, widely used in the literature, should be sufficient to understand the realistic 2-sphere model used in chapter \ref{ch:new_model}.

%Chapter \ref{ch:forward} shows the derivation for the forward model of Electrical Source Imaging.
%
%At some point, cite \cite{grech2008review}.

Chapter \ref{ch:review} presents a basic review of the literature on ESI methods, parameter tuning, and performance evaluation of ESI methods.

%Despite being relatively small, the selection of algorithms is discussed orderly

This chapter presents some assumptions that have been incorporated into the ESI model, leading to algorithms with improved efficiency.
%
Some of these ideas will be used to construct a novel method reflecting specific assumptions.

The parameter tuning and performance metrics methods will also be used in chapters \ref{ch:numeric} and \ref{ch:new_model}.

Chapter \ref{ch:numeric} describes a protocol for creating synthetic data and explores some characteristics induced into the resulting dataset.

Chapter \ref{ch:new_model} describes a novel ESI method based on anatomical data provided by observed symptoms.

The model is derived from the assumptions, and we propose a way to implement it efficiently.
%
This should prove that the proposed method isn't more computationally expensive than popular fast methods, such as sLORETA.

The proposed method is later used on a real dataset from an induced stroke in an animal model.
%
The characteristics of the real data are the motivation of the model assumptions.

