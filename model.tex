\chapter{Proposed Model}
%
%%%%%%%%%%%%%%%%%%%%%%%%%%%%%%%%%%%%%%%%%%%%%%%%%%%%%%%%%
%%%%%%%%%%%%%%%%%%%                                                   %%%%%%%%%%%%%%%%%%%%%%
%%%%%%%%%%%%%%%%%%%                                                   %%%%%%%%%%%%%%%%%%%%%%
%%%%%%%%%%%%%%%%%%%%%%%%%%%%%%%%%%%%%%%%%%%%%%%%%%%%%%%%%
%%%%%%%%%%%%%%%%%%%%%%%%%%%%%%%%%%%%%%%%%%%%%%%%%%%%%%%%%%%%%%%%%%%%%%%%%%
%%%%%%%%%%%%%%%%%%                                             Stability of multipeakons                                            %%%%%%%%%%%%%%%%
%%%%%%%%%%%%%%%%%%%%%%%%%%%%%%%%%%%%%%%%%%%%%%%%%%%%%%%%%%%%%%%%%%%%%%%%%%
%
%

Within the asymmetric multi-modal data analysis paradigm, we aim to enhance the ESI data by incorporating information from other measurement modalities.

In particular, we consider the case in which a physical region is known to produce a higher background electrical activity.
%
The location and extent of this region are observed using some imaging techniques independent of EEG.

The motivation for this particular setup comes from a specific experimental setup in which an ischemic stroke is induced, and later, the affected area is determined using histochemical analysis.

\section{Model Assumptions}

The relationship between the recordings from surface electrodes, $\Y$, and the magnitudes of equivalent distributed dipoles located inside the brain, $\SA$, is given by the following equation
\begin{equation}
\Y = \G \ppar{\SA + \varepsilon},
\label{eq:3.1}
\end{equation}
with $\Y \in \R^{M \times 1}$, $\SA \in \R^{3N \times 1}$, $\G \in \R^{3N\times M}$, and $\varepsilon\in \R^{N\times 1}$.
%
This model was described in detail in chapter [], including the interpretation and derivation of the leadfield matrix $\G$.
%
Notice that this model considers only internal noise.

Ideally, the extra information provided by the additional data modality should allow us to identify an anatomical region exhibiting pathological behavior; for simplicity, we refer to this region as a P-region. 
%
For generality, we consider the possibility of multiple disjoint P-regions, say $K$ with $1\leq K < \infty$.
%
P-regions are labeled $P_1, P_2, \dots, P_K$, with $P_0$ the dipoles outside all P-regions.
%
The P-regions regions are encoded into the model using a matrix 
$\PA\in \sset{0,1}^{N\times K}$ defined as
\begin{equation}
    \PA\ppar{n,k} = \begin{cases}
        1, &\text{if } n \in P_k \\
        0, &\text{otherwise.}
    \end{cases}
\end{equation}

In section [], we describe heuristic methods to determine the P-regions and, consequently, the matrix $\PA$.

%The P-region is encoded into the model as a labeling variable 
%${Z\in \sset{0,1}^{N\times 1}}$, so that $Z_n = 1$ if the $n$-th dipole is located on a the active region.

%\begin{figure}
%\centering
%\includegraphics[width=0.8\linewidth]{./img/sketch02_v2}
%\end{figure}

The relation between the current density, $\SA$, and the P-regions is incorporated into the model by adjusting the covariance of $\SA$ depending on which P-region it is located.
%
To be specific, consider,
\begin{equation}
    \ccov{\SA\ppar{n_1}, \SA\ppar{n_2}} = 
    \begin{cases}
        \gamma_k, & \text{if } n_1, n_2\in P_k, \\
        0, & \text{otherwise}
    \end{cases}
\end{equation}
with $\gamma_0 \ll \gamma_k$ for $1\leq k \leq K$.
%
This guarantees that P-regions have synchronized activity that is larger in magnitude than the background activity.

The assumption of a perfect synchrony is introduced in order to simplify the model.
%
We can separate the source, $\SA$, into a regionally-synchronised activity, $\RA$, and a background noise, $\BA$, as
\begin{equation}
    \SA +\varepsilon = \BA + \PA \RA = \BA + \sum_{k=1}^K \PA(:,k)\, \RA(k)
    \label{eq:3.4}
\end{equation}
with $\BA \in \R^{N\times 1}, \RA \in \R^{K\times 1}$ independent with each other.
%
These assumptions about P-regions can be combined with Gaussian assumptions, leading to 
\begin{align}
    \BA &\sim \norm\ppar{0, \gamma_0 \id_N } 
    \label{eq:3.5} \\
    \RA &\sim \norm\ppar{0, \text{diag}\ppar{\gamma_1, \gamma_2, \dots, \gamma_K}-\gamma_0 \id_N }
    \label{eq:3.6}
\end{align}

\section{Proposed Estimator}

The proposed estimator is inspired by the concept of Maximum A Posteriori Estimators described in Chapter 2.
%
This choice follows the analysis of this type of estimator's strengths and weaknesses, balancing towards shorter computation times and memory requirements.
%
We expect that the additional information from P-regions will increase the low accuracy of this type of estimator.

The estimator of $\SA$ will be constructed as 
\begin{equation}
    \hat{\SA} = \hat{\BA} + \PA \hat{\RA}
\end{equation}
where $\hat{\BA}$ and $\hat{\RA}$ are solutions of the following minimization problem
\begin{align}
    \ppar{ \hat{\BA}, \hat{\RA} } 
    =
    \argmax_{\BA, \RA }\, 
    &\phantom{.}
    \frac{1}{2 \theta_0} \nnorm{\BA}^2
    +
    \sum_{k=1}^K \frac{1}{2 \theta_0} \nnorm{\PA(:,k)\, \RA(k)}^2
    \nonumber \\
    \text{s.t.}
    \quad
    &\phantom{.}
    {\G \ppar{\BA + \PA \RA} } - \Y = 0
    \label{eq:3.7}
\end{align}

A careful derivation will reveal that $\theta_0 = \frac{\gamma_0 }{\sigma}$ and $\theta_k = \frac{\gamma_k-\gamma_0}{\sigma}$, with $\sigma>0$ the variance of the external noise.
%
%However, we decided that the role of those parameters within the interpretability of the model is negligible; we use $\theta_k$ as generic parameters that need to be tuned using the heuristics discussed in Chapter 2.

The constrained optimization problem in equation \eqref{eq:3.7} is solved using the method Lagrange multipliers.
%
First, we construct the respective lagrangian function
\begin{align}
    \mathcal{L}\ppar{\BA, \RA, \lambda} &=
    \frac{1}{2 \theta_0} \nnorm{\BA}^2
    +
    \sum_{k=1}^K \frac{1}{2 \theta_k} \nnorm{\PA(:,k)\, \RA(k)}^2
    +
    \lambda^T \ppar{{\G \ppar{\BA + \PA \RA} } - \Y}
\end{align}
which can be minimized by solving the normal equations, i.e., the partial derivatives equal to zero.
%
The partial derivatives are given by
\begin{align}
    \frac{\partial}{\partial \BA} \mathcal{L}\ppar{\BA, \RA, \lambda}
    &=
    \frac{1}{\theta_0} \BA + \G^T \lambda 
    \label{eq:3.9}
    \\
    \frac{\partial}{\partial \RA(k)} \mathcal{L}\ppar{\BA, \RA, \lambda}
    &=
    \frac{1}{\theta_k} \PA(:,k)^T \PA(:,k)\, \RA(k) + \PA(:,k)^T \G^T \lambda 
    \label{eq:3.10}
    \\
    \frac{\partial}{\partial \lambda} \mathcal{L}\ppar{\BA, \RA, \lambda}
    &=
    \G \ppar{ \BA + \PA \RA } - \Y
    \label{eq:3.11}
\end{align}

Solving the normal equation for \eqref{eq:3.9} and \eqref{eq:3.10} lead to the following identities
\begin{align}
    \BA &= -\theta_0 \G^T \lambda
    \\
    \RA(k)
    &=
    -\theta_k \spar{\PA(:,k)^T \PA(:,k)}^{-1} \PA(:,k)^T \G^T \lambda 
\end{align}

Notice that $\PA(:,k)^T \PA(:,k) \in \mathbb{N}^{1\times 1}$ represents the number of dipoles on the $k$-th P-region, and thus is invertible.
%
For ease of notation, define
\begin{equation}
    \abss{P_k} = \PA(:,k)^T \PA(:,k)
\end{equation}

From equation \eqref{eq:3.11}, we may obtain a closed-form expression for $\lambda$,
\begin{align}
    \Y
    &=
    \G \BA + \G { \sum_{k=1}^K { \PA(:,k)\, \RA(k) } }
    \nonumber \\
    &=
    \G\ppar{-\theta_0 \G^T \lambda}
    + \G \sum_{k=1}^K
    { \PA(:,k)\, \ppar{-\theta_k \abss{P_k}^{-1} \PA(:,k)^T \G^T \lambda}}
    \nonumber \\
    &=
    -\G \ppar{ \theta_0 \id_N + \sum_{k=1}^K \theta_k \abss{P_k}^{-1} 
    \PA(:,k)\, \PA(:,k)^T
    } \G^T \lambda
\end{align}
which leads to the following identity
\begin{equation}
    \lambda
    =
    -\spar{\G \ppar{ \theta_0 \id_N + \sum_{k=1}^K \theta_k \abss{P_k}^{-1} 
    \PA(:,k)\, \PA(:,k)^T
    } \G^T}^{-1} \Y
\end{equation}

For robustness, we propose using a regularized $\lambda_\rho$ defined as
\begin{equation}
    \lambda_\rho
    =
    -\spar{\G \ppar{ \theta_0 \id_N + \sum_{k=1}^K \theta_k \abss{P_k}^{-1} 
    \PA(:,k)\, \PA(:,k)^T
    } \G^T + \rho \id_M }^{-1} \Y
\end{equation}
and for a further simplification of notation, define
\begin{align}
    W_\rho &=
    \spar{\G \ppar{ \theta_0 \id_N + \sum_{k=1}^K \theta_k \abss{P_k}^{-1} 
    \PA(:,k)\, \PA(:,k)^T
    } \G^T + \rho \id_M }^{-1}
    \\
    \lambda_\rho &= -W_\rho  \Y
\end{align}

Using $\lambda_\rho$ 




 {Maximum A Posteriori (MAP) estimator}
    The iterative minimization of the error function can be interpreted as a multi-scale iterative correction
    \begin{align}
        \hat{\N}^{(i+1)} &= \argmin_{\N} F\ppar{\U^{(i)},\N; \gamma_0, \gamma_1}
        \\
        \hat{\U}^{(i+1)} &= \argmin_{\U} F\ppar{\U,\N^{(i)}; \gamma_0, \gamma_1}
    \end{align}
    These steps have the following closed-form
    \begin{align}
        \hat{\N}^{(i+1)} &=
        \hat{\N}^{(i)}
        -
        \G^T \spar{\G \G^T + \frac{\gamma_0^2}{\sigma^2} \id}^{-1} L_S \ppar{\hat{\U}^{(i)}-\hat{\U}^{(i-1)} }
        \\
        \hat{\U}^{(i+1)} &=
        \hat{\U}^{(i)}
        -
        L_S^T \G^T \spar{\G L_S L_S^T \G^T + \frac{\gamma_1^2}{\sigma^2} \id}^{-1} \ppar{\hat{\N}^{(i)}-\hat{\N}^{(i-1)} }
    \end{align}

\section{Results}

\subsection{Real Data}

The effectiveness of the method is tested using data from an experiment 
of acute ischemic stroke on an animal model, performed by
Pascal [cite].

Posterior to the experiment, the subject is sacrificed.
%
The brain is fragmented in frontal slices for staining with
2, 3, 5 
triphenyltetrazolium (TTC)
to identify the anatomical region damaged by hypoxia\footnote{Lack of oxygen at a cellular level.}, which is also known as the Ischemic Penumbra

The operation of the TTC staining is based on the fact that TTC (white) 
%is a white compound, but it 
is degraded to 1,3,5-triphenyl formazan (TPF, red)
on the presence of dehydrogenases in metabolically active cells.
%
As a result, tissue colored white was affected by necrosis, and tissue colored red was unaffected. 
[cite Wiki?]

%\begin{figure}
%\centering
%\includegraphics[width=0.8\linewidth]{./img/sketch01_v2}
%%\caption{This figure is temporary and will be replaced}
%\end{figure}

Within the framework of this work, pictures obtained after TTC staining were registered to the template MRI in order to identify the Ischemic Penumbra at the time of sacrifice.
%
This definition is simplistic and doesn't represent a clinical decision.

The active region defined in the section [], $Z$, is determined using the obtained information: a dipole $i$ is labeled as $Z_i=1$ if and only if it is located on the Ischemic Penumbra.



\section{Previous text}





 



 




 


