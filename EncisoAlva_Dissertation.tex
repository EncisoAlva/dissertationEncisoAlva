%%%%%%%%%%%%%%%%%%%%%%%%%%%%%%%%%%%%%%%%%%%%%%%%%%%%%%%%%%%%%%%%%%%%%%%%%%%%%%%%%%
%%%%  UTA Ph.D. Dissertation Document Generation Using Tex/Latex   %%%%%%%%%%%%%%%%%%%
%%%%%%%%%%%%%%%%%%%%%%%%%%%%%%%%%%%%%%%%%%%%%%%%%%%%%%%%%%%%%%%%%%%%%%%%%%%%%%%%%%
%     Tex/Latex is one of the widely used software to generate many technical and other
%     documents including thesis/dissertations at many universities including UTA.
%
%     The advantage of using Tex/Latex is that with proper style files, most of the
%     front matter required by the university is generated with very little
%     effort (including table of contents, copyright page, list of figures, etc.).
%     All other formatting (margins, fonts, footnote style, equation style,
%     table style, etc.) is built into the style file. With proper style files,
%     it is possible to convert many of the material in any Tex/Latex document
%     so that the generated output fulfills the requirements of an outside publication.
%     Many of the outside publications provide such style files for the conversion.
%
%     The utathesis.zip contains all the sample files that are necessary to generate
%     a typical UTA Ph.D. Dissertation Document including utathesis.sty (contains
%     all the formatting required by UTA Graduate School), graphics.sty, psfig.sty
%     (to include figures that may be in the thesis), amsmath.sty (helpful in generating
%     more complex equations), two typical chapters (which contain equations, figures, and
%     tables, references), two appendices, typical dedication, acknowledgement, abstract,
%     and biographic information files. It is suggested that you unzip utathesis.zip
%     into a new directory so that you can run utaexample.tex contained in
%     the directory to generate a sample output. The utaexample.tex is like a script file
%     which you can modify to suit individual requirements and any minor changes that
%     are required. By commenting out certain portions of the utaexample.tex, it is
%     possible to generate truncated output (just one chapter without front matter, etc.).
%
%     Any Tex/Latex Package including MiKTex (http://www.miktex.org)
%     and Convenience Tex/Latex Editors (http://www.miktex.org/links.html) may be used
%     to generate the Dissertation Document.
%
%     This utaexample.tex file was created by UTA EE Department and uses the
%     bibliography/reference style used by IEEE and adopted by EE Department.
%     The bibliography/reference file acceptable to other departments at UTA are available
%     and must be substituted in the appropriate place.
%
%     Any comments/suggestions  you may have may be sent to prabhu@uta.edu.
%
%%%%%%%%%%%%%%%%%%%%%%%%%%%%%%%%%%%%%%%%%%%%%%%%%%%%%%%%%%%%%%%%%%%%%%%%%%%%%%%%%%

%%%%%%%%%%%%%%%%%%%%%%%%%%%%%%%%%%%%%%%%%%%%%%%%%%%%%%%%%%%%%%%%%%%%%%%%%%%%%%%%%%
%%%%  UTA Ph.D. Dissertation Document Generation Using Tex/Latex   %%%%%%%%%%%%%%%%%%%
%%%%%%%%%%%%%%%%%%%%%%%%%%%%%%%%%%%%%%%%%%%%%%%%%%%%%%%%%%%%%%%%%%%%%%%%%%%%%%%%%%

%%%%%%%%%%%%%%%%%%%%%%%%%%%%%%%%%%%%%%%%%%%%%%%%%%%%%%%%%%%%%%%%%%%%%%%%%%%%%%%%%%
%%%%%%%%%%%%%%%%         all the preamble material            %%%%%%%%%%%%%%%%%%%%
%%%%%%%%%%%%%%%%%%%%%%%%%%%%%%%%%%%%%%%%%%%%%%%%%%%%%%%%%%%%%%%%%%%%%%%%%%%%%%%%%%

% packages
\documentclass[12pt]{report}\usepackage{utathesis,amsmath,amsthm,amssymb,listings,color,graphicx,float,enumitem,algorithm,algpseudocode,oldgerm,mathrsfs,microtype,geometry,latexsym,xpatch,etoolbox,tikz,bm}%,showkeys}
\usepackage[english]{babel}
\usepackage[utf8x]{inputenc}
\usepackage[style=1]{mdframed}

% for units
\usepackage{siunitx}

% for the vector on ones
\usepackage{bbm}

% Sample bibliography materials
% Entries found here: http://ctan.math.utah.edu/ctan/tex-archive/macros/latex/contrib/biblatex/bibtex/bib/biblatex/biblatex-examples.bib
%\usepackage{biblatex}


% A nice way to customize commands, e.g., \C instead of \mathfrak{C} for the set of complex numbers
% Suggestion: name macros specific to document: macros_diss, macros_colloquium, etc.
\input macros_disstemplate.tex

    \begin{document}
    \input{psfig.sty}
       % As of 2020, graduation months must be May or December
       \graduationmonth{May}
       % Congratulations for this year!
       \graduationyear{2024 ?}
       % Always check valid defense dates
       % See Graduation Deadlines on the site of the UTA Registrar
       % Spring 2020 Deadlines: defense, May 1; dissertation, May 8.
       \defensedate{???}
       \author{JJulio C Enciso-Alva}
       % End with "{}{}" if only 4 committee members; end with "{}" if only 5 committee members, i.e., delete text within brackets but keep brackets
       \committee{SupervisingProfessor}{MemberA}{MemberB}{MemberC}{MemberD}{MemberE}
       \title{Title}


%%%%%%%%%%%%%%%%%%%%%%%%%%%%%%%%%%%%%%%%%%%%%%%%%%%%%%%%%%%%%%%%
%%%%%%%%%%%%%%  title page  %%%%%%%%%%%%%%%%%%%%%%%%%%%%%%%%
%%%%%%%%%%%%%%%%%%%%%%%%%%%%%%%%%%%%%%%%%%%%%%%%%%%%%%%%%%%%%%%%

          \titlepage


%%%%%%%%%%%%%%%%%%%%%%%%%%%%%%%%%%%%%%%%%%%%%%%%%%%%%%%%%%%%%%%%
%%%%%%%%%%%%%%  copyright page  %%%%%%%%%%%%%%%%%%%%%%%%%%%%%%%%
%%%%%%%%%%%%%%%%%%%%%%%%%%%%%%%%%%%%%%%%%%%%%%%%%%%%%%%%%%%%%%%%

         \copyrightpage

%%%%%%%%%%%%%%%%%%%%%%%%%%%%%%%%%%%%%%%%%%%%%%%%%%%%%%%%%%%%%%%%
%%%%%%%%%%%%%%  Dedication page  %%%%%%%%%%%%%%%%%%%%%%%%%%%%%%%%
%%%%%%%%%%%%%%%%%%%%%%%%%%%%%%%%%%%%%%%%%%%%%%%%%%%%%%%%%%%%%%%%


%%%%%%%%%%%%%%%%%%%%%%%%%%%%%%%%%%%%%%%%%%%%%%%%%%%%%%%%%%%%%%%
%%%%%%%%%%%%%%  acknowledgements  %%%%%%%%%%%%%%%%%%%%%%%%%%%%%%%%
%%%%%%%%%%%%%%%%%%%%%%%%%%%%%%%%%%%%%%%%%%%%%%%%%%%%%%%%%%%%%%%%
\newpage

\begin{acknowledgements}

\input acknowledge.tex

\end{acknowledgements}
%%%%%%%%%%%%%%%%%%%%%%%%%%%%%%%%%%%%%%%%%%%%%%%%%%%%%%%%%%%%%%%
%%%%%%%%%%%%%%     abstract    %%%%%%%%%%%%%%%%%%%%%%%%%%%%%%%%
%%%%%%%%%%%%%%%%%%%%%%%%%%%%%%%%%%%%%%%%%%%%%%%%%%%%%%%%%%%%%%%%
\begin{abstract}
	\input abstract.tex
	\indent
\end{abstract}

\tableofcontents
\addtocontents{toc}{\noindent\mbox{Chapter}\hfill\mbox{Page}}%
%\addtocontents{toc}{\noindent\mbox{Chapter}}%
%%%%%%%%%%%%%%%%%%%%%%%%%%%%%%%%%%%%%%%%%%%%%%%%%%%%%%%%%%%%%%%
%%%%%%%%%%%%%%  First and Following Chapters  %%%%%%%%%%%%%%%%%%
%%%%%%%%%%%%%%%%%%%%%%%%%%%%%%%%%%%%%%%%%%%%%%%%%%%%%%%%%%%%%%%%
%     \input macros.tex      % file containing author's macro definitions
\input intro.tex       % file containing Chapter 1 contents
\input literature.tex  % file containing Chapter 2 contents
%\input model.tex      % file containing Chapter 3 contents
%\input virt_elec.tex      % file containing Chapter 4 contents
%%%%%%%%%%%%%%%%%%%%%%%%%%%%%%%%%%%%%%%%%%%%%%%%%%%%%%%%%%%%%%%%%%%%%%%%%%%%%%%%%%
%%%%%%%%%%%%%%%%                  Appendices                  %%%%%%%%%%%%%%%%%%%%
%%%%%%%%%%%%%%%%%%%%%%%%%%%%%%%%%%%%%%%%%%%%%%%%%%%%%%%%%%%%%%%%%%%%%%%%%%%%%%%%%%
%\appendix
%      \chapter{FIRST APPENDIX NAME}
%\chapter{INEQUALITY}
%\input appA.tex       % file with Appendix A contents
%\chapter{UPPER BOUNDS}
%\input appB.tex       % file with Appendix B contents
%%%%%%%%%%%%%%%%%%%%%%%%%%%%%%%%%%%%%%%%%%%%%%%%%%%%%%%%%%%%%%%%%%%%%%%%%%%%%%%%%%
%%%%%%%%%%%%%%%%                 Bibliography                 %%%%%%%%%%%%%%%%%%%%
%%%%%%%%%%%%%%%%%%%%%%%%%%%%%%%%%%%%%%%%%%%%%%%%%%%%%%%%%%%%%%%%%%%%%%%%%%%%%%%%%%

\bibliographystyle{abbrv} % We choose the "plain" reference style
\bibliography{./refs} % Entries are in the refs.bib file

%\renewcommand{\bibname}{REFERENCES}
%\bibliographystyle{plain}
%\nocite{*}      
%\input{mybib.tex}


%%%%%%%%%%%%%%%%%%%%%%%%%%%%%%%%%%%%%%%%%%%%%%%%%%%%%%%%%%%%%%%%%%%%%%%%%%%%%%%%%%
%%%%%%%%%%%%%%%%         Biographical Statement               %%%%%%%%%%%%%%%%%%%%
%%%%%%%%%%%%%%%%%%%%%%%%%%%%%%%%%%%%%%%%%%%%%%%%%%%%%%%%%%%%%%%%%%%%%%%%%%%%%%%%%%
\thebiography
\input biography.tex
\end{document}
