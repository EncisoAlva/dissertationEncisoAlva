A central task for neuroscience is to determine the location of electrical activity inside the brain. Such electrical signals can be recorded at a high resolution in time (sub-millisecond) but low resolution in space, thus making it difficult to locate its source unambiguously. Further assumptions must be incorporated into the electrical source models to locate this electrical activity inside the brain reliably. One such assumption is to consider the current density distribution and assume that, among all possible configurations, the ones with minimal energy are more likely to be correct. The specifics of implementing this assumption have led to a multitude of methods. However, these minimal-norm methods are limited in terms of the quality of electrical recordings and their low resolution in space. In the paradigm of multi-modal data fusion, the electrical source localization methods are enhanced by considering data from additional imaging modalities. 

This work proposes a simple model using binarized pathology data to enhance electrical source imaging from electroencephalography (EEG) recordings. It is motivated by post-mortem data on hypoxia due to ischemic stroke, but it may use data derived from fMRI, NIRS, and CT, among others.